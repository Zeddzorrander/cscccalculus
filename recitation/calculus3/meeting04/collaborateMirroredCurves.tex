\documentclass[handout,noauthor,nooutcomes]{ximera}

\author{Bart Snapp}

\usepackage{multicol}

\outcome{}

\input{../../../preamble.tex}

\title[Collaborate:]{Mirrored curves}

\begin{document}
\begin{abstract}
  We study how to bounce vectors off of curves.
\end{abstract}
\maketitle

\textbf{Work in groups of 3--4, writing your answers on a separate
  sheet of paper.}

\section{Normal vectors}

\begin{problem}
Consider the following line:
\begin{image}
  \begin{tikzpicture}
    \begin{axis}[
        xmin=-1,xmax=5,ymin=-1,ymax=3,
        clip=true,
        axis lines=center,
        %ticks=none,
        unit vector ratio*=1 1 1,
        xlabel=$x$, ylabel=$y$,
        %ytick={-2,-1,...,7},
	%xtick={-2,-1,...,10},
	grid = major,
        every axis y label/.style={at=(current axis.above origin),anchor=south},
        every axis x label/.style={at=(current axis.right of origin),anchor=west},
      ]
      \addplot[ultra thick,penColor] {-x/2+2};
      %\addplot[very thick,penColor2,->] plot coordinates {(2,1) (3,3)};
    \end{axis}
  \end{tikzpicture}
\end{image}
Find a vector normal to this line. Explain your reasoning.
\end{problem}

\begin{problem}
  Now consider the line $ax+by = c$ in $\R^2$. Find a vector normal to
  this line. Explain your reasoning.
\end{problem}

\begin{problem}
  Consider the equation:
  \[
  \vec{n}\dotp (\vec{x}-\vec{p}) = 0
  \]
  Explain how this connects to finding normal vectors to lines in
  $\R^2$ of the form:
  \[
  ax + by = c
  \]
  In particular, you should explain what $\vec{n}$, $\vec{x}$, and
  $\vec{p}$ represent.
\end{problem}


\begin{problem}
  Quick! Tell me normal vectors for the following lines:
  \begin{multicols}{2}
  \begin{enumerate}
  \item $-3x+7y=11$
  \item $4y =8$
  \item $x=y$
  \item $y=-4x+1$
  \end{enumerate}
  \end{multicols}
\end{problem}


\section{Reflecting off of lines}


Now we will explore how mirrors reflects light.

\begin{fact}[Law of Reflection]
  Light is reflected at the same angle as it arrived, as
  measured from a line perpendicular to the mirror. Draw a picture
  illustrating this fact. 
\end{fact}

Let's see if we can explain why the Law of Reflection is true. We'll
address this in the next several problems.


\begin{problem}
  Consider the following diagram:
  \begin{image}
    \begin{tikzpicture}
      \coordinate (A) at (0,2);
      \coordinate (B) at (0,5);
      \coordinate (C) at (8,1);
      \coordinate (E) at (8,4);
      \coordinate (D) at (4,3);
      
      \draw[ultra thick,penColor] (A)--(E)--(D);
      \draw[thick,penColor3,->] (B)--(D);
      \draw[dashed] (D)--(C);
      \tkzMarkAngle[size=0.7cm,thin](B,D,A)
      \tkzLabelAngle[pos = -0.4](B,D,A){$\alpha$}
            
      \tkzMarkAngle[size=0.9cm,thin](C,D,E)
      \tkzLabelAngle[pos = 0.6](C,D,E){$\beta$}
            
      %\draw[step=.5cm] (0,0) grid (10,5);
    \end{tikzpicture}
  \end{image}
  Explain why the opposite angles $\alpha$ and $\beta$ must be
  equal.
  \begin{hint}
    Label more angles in your picture. Some pairs of angles will sum
    to $180^\circ$. Use this to conclude that $\alpha = \beta$.
  \end{hint}
\end{problem}

Since a mirror simply reflects light, we see that the initial light
beam makes the same angle with the line as the reflected light beam:
\begin{image}
  \begin{tikzpicture}
      \coordinate (A) at (0,2);
      \coordinate (B) at (0,5);
      \coordinate (C) at (8,1);
      \coordinate (E) at (8,4);
      \coordinate (D) at (4,3);
      \coordinate (CR) at (6.59,6.65);
      
      \draw[ultra thick,penColor] (A)--(E)--(D);
      \draw[thick,penColor3,->] (B)--(D);
      \draw[thick,penColor3,->,dashed] (D)--(CR);
      \draw[dashed] (D)--(C);
      \tkzMarkAngle[size=0.7cm,thin](B,D,A)
      \tkzLabelAngle[pos = -0.4](B,D,A){$\alpha$}

      \tkzMarkAngle[size=0.9cm,thin](C,D,E)
      \tkzLabelAngle[pos = 0.6](C,D,E){$\alpha$}
      
      \tkzMarkAngle[size=0.7cm,thin](E,D,CR)
      \tkzLabelAngle[pos = 0.4](E,D,CR){$\alpha$}
            
      
      %\draw[step=.5cm] (0,0) grid (10,5);
  \end{tikzpicture}
\end{image}

\begin{problem}
  Adding a normal vector to the diagram above:
  \begin{image}
    \begin{tikzpicture}
      \coordinate (A) at (0,2);
      \coordinate (B) at (0,5);
      \coordinate (C) at (8,1);
      \coordinate (E) at (8,4);
      \coordinate (D) at (4,3);
      \coordinate (CR) at (6.59,6.65);
      \coordinate (F) at (3,7);
      
      \draw[ultra thick,penColor] (A)--(E)--(D);
      \draw[thick,penColor3,->] (B)--(D);
      \draw[thick,penColor3,->,dashed] (D)--(CR);
      \draw[thick,penColor2,->] (D)--(F);
      %% \tkzMarkAngle[size=0.7cm,thin](B,D,A)
      %% \tkzLabelAngle[pos = -0.4](B,D,A){$\alpha$}

      %\tkzMarkAngle[size=0.9cm,thin](C,D,E)
      %\tkzLabelAngle[pos = 0.6](C,D,E){$\alpha$}
      
      %% \tkzMarkAngle[size=0.7cm,thin](E,D,CR)
      %% \tkzLabelAngle[pos = 0.4](E,D,CR){$\alpha$}

      \tkzMarkAngle[size=0.7cm,thin](CR,D,F)
      \tkzLabelAngle[pos = 0.4](CR,D,F){$\theta$}
      
      \tkzMarkAngle[size=0.8cm,thin](F,D,B)
      \tkzLabelAngle[pos = 0.5](F,D,B){$\varphi$}
            
            
      %\draw[step=.5cm] (0,0) grid (10,5);
    \end{tikzpicture}
  \end{image}
  Explain why $\theta$ and $\varphi$ are equal.
\end{problem}


\begin{problem}
  Someone says that if $\vec{n}$ is a normal vector to a mirror, and
  $\vec{v}$ represents a light beam, then the reflected light beam is
  given by:
  \[
  \vec{r}=\vec{v} - 2\proj_{\vec{n}}(\vec{v})
  \]
  Give \textbf{two} explanations verifying the formula above:
  \begin{itemize}
  \item An explanation where you sketch $\vec{r}$ and show that it is
    a reasonable answer.
  \item An explanation where you show (via a computation) that the
    angle which light is reflected is the same angle as it arrived, as
    measured from a line perpendicular to the mirror.
  \end{itemize}
\end{problem}




\begin{problem}
  Now imagine the line below as a mirror and imagine vector $\vec{v}$
  as a light-ray that strikes the mirror. Find the vector representing
  the reflection of $\vec{v}$.
  \begin{image}
  \begin{tikzpicture}
    \begin{axis}[
        xmin=-1,xmax=5,ymin=-1,ymax=3,
        clip=true,
        axis lines=center,
        %ticks=none,
        unit vector ratio*=1 1 1,
        xlabel=$x$, ylabel=$y$,
        %ytick={-2,-1,...,7},
	%xtick={-2,-1,...,10},
	grid = major,
        every axis y label/.style={at=(current axis.above origin),anchor=south},
        every axis x label/.style={at=(current axis.right of origin),anchor=west},
      ]
      \addplot[ultra thick,penColor] {-x/2+2};
      %\addplot[very thick,penColor2,->] plot coordinates {(2,1) (3,3)};
      \addplot[ultra thick,penColor3,->] plot coordinates {(1,3) (2,1)};
      %\addplot[very thick,dashed,penColor3,->] plot coordinates {(2,1) (4,1)};
    \end{axis}
  \end{tikzpicture}
\end{image}
\end{problem}


\section{Reflecting off of curves}

Reflecting off of mirrored surfaces doesn't require calculus unless
the mirror is curved. 

\begin{problem}
  Let
  \[
  \vec{p}(t) = \vector{t,t^2}
  \]
  We claim that $\vec{n} = \vector{2t,-1}$ is a normal vector for this
  curve for any given $t$. Confirm or deny this claim.
\end{problem}

\begin{problem}
  Suppose light rays $\vec{v}=\vector{0,-1}$ are hitting a mirrored
  surface described by:
  \[
  \vec{p}(t) = \vector{t,t^2}
  \]
  Find a formula for the vector reflected off the surface for any
  given value of $t$. Call this vector $\vec{r}(t)$. Explain your
  reasoning.
\end{problem}


\begin{problem}
  For any given value of $t$, $\vector{t,t^2}$ is a point on our
  curve. In the previous problem you found a vector $\vec{r}(t)$
  describing a reflected light beams $\vec{v}=\vector{0,-1}$. Consider
  the line:
  \[
  \vecl(s) = \vector{t,t^2} + s \cdot \vec{r}(t)
  \]
  \begin{itemize}
  \item Find $a$ so that $\vecl(a)$ intersects the $y$-axis.
  \item What is the $y$-value of $\vecl(a)$?
  \item Do the questions directly above tell us? What does this have
    to do with telescopes, space-heaters, vanity mirrors, and
    eavesdropping devices?
  \end{itemize}
\end{problem}

\end{document}
