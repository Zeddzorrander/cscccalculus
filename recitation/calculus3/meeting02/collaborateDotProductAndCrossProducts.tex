\documentclass[handout,nooutcomes,noauthor]{ximera}

\author{Bart Snapp}

\input{../../../preamble.tex}

\title[Collaborate:]{Dot products and cross products}

\begin{document}
\begin{abstract}
  Working with the dot product and cross product. 
\end{abstract}
\maketitle

\textbf{Work in groups of 3--4, writing your answers on a separate
  sheet of paper.}

\begin{problem}
  Draw and label two vectors $\vec{v}$ and $\vec{w}$ such that
  \begin{enumerate}
  \item $\vec{v}\dotp\vec{w} <0$
  \item $\vec{v}\dotp\vec{w} =0$
  \item $\vec{v}\dotp\vec{w} >0$
  \end{enumerate}
\end{problem}

\begin{problem}
  Give \textbf{three} explanations (one algebraic, one geometric, one
  using trigonometry) why
  \[
  \proj_{\vec{a}}(\vec{b}) = \vec{0}
  \]
  if $\vec{a}\dotp\vec{b} = 0$.
\end{problem}

\begin{problem}
  Give two explanations (one algebric and one geometric) that shows
  that
  \[
  \vec{b} - \proj_{\vec{a}}(\vec{b}) \quad\text{is orthogonal to}\quad \vec{a}
  \]
\end{problem}


\begin{problem}
  Let $\vec{r}$ be a random (nonzero) vector in $\R^2$. Compute:
  \[
  \proj_\vecj\left(\proj_\veci(\vec{r})\right)
  \]
  Explain your reasoning.
\end{problem}

\begin{problem}
  Let $\vec{r}$ be a random (nonzero) vector in $\R^2$. Could there be
  a vector $\vec{v}$ such that one expects
  \[
  \proj_\vecj\left(\proj_{\vec{v}}\left(\proj_\veci(\vec{r})\right)\right)\ne \vec{0}?
  \]
  Explain your reasoning.
\end{problem}





\begin{problem}
  Consider three vectors in $\R^3$:
  \begin{align*}
    \vec{a} &= \vector{a_1,a_2,a_3}\\
    \vec{b} &= \vector{b_1,b_2,b_3}\\
    \vec{c} &= \vector{c_1,c_2,c_3}
  \end{align*}
    We claim:
    \[
    \det \begin{bmatrix}
      a_1 & a_2 & a_3\\
      b_1 & b_2 & b_3\\
      c_1 & c_2 & c_3
    \end{bmatrix}
    =
      \vec{a}\dotp(\vec{b}\cross\vec{c}) =
    (\vec{a}\cross\vec{b})\dotp\vec{c}
    \]
    Give an algebraic verification of this claim.  For your
    information, $(\vec{a}\cross\vec{b})\dotp \vec{c}$ is commonly
    called the \dfn{scalar triple product}.
 \end{problem}


\begin{problem}
  Given vectors $\vec{a}$, $\vec{b}$, and $\vec{c}$, in $\R^3$, we
    claim:
    \[
    \vec{a}\dotp(\vec{b}\cross\vec{c}) =
    (\vec{a}\cross\vec{b})\dotp\vec{c}
    \]
    Use the diagram
    \begin{image}
    \begin{tikzpicture}
      \coordinate (A) at (0,0);
      \coordinate (B) at (4,-1);
      \coordinate (C) at (4,1); 
      \coordinate (D) at (0,4);
      \coordinate (E) at (2,3);
      \coordinate (F) at (8,0);
      \coordinate (G) at (6,2);
      \coordinate (H) at (6,4);
      \coordinate (I) at (10,3);


      \draw[draw=none,pattern=north west lines, pattern color=penColor3] (A)--(B)--(F)--(C)--cycle;
      %% \draw[dashed,penColor] (O)--(E);
      %% \draw[dashed,penColor2] (O)--(F);
      %% \draw[dashed,penColor4] (O)--(D); 

      \draw[->,ultra thick,penColor,shorten <=2pt,shorten >=5pt] (A)--(B);
      \draw[->,ultra thick,penColor2,shorten <=2pt,shorten >=5pt] (A)--(C);
      \draw[->,ultra thick,penColor4,shorten <=2pt,shorten >=5pt] (A)--(E);
      \draw[->,ultra thick,penColor3,shorten <=2pt,shorten >=5pt] (A)--(D);

      \draw[ultra thick, gray] (B)--(G);
      \draw[ultra thick, gray] (B)--(F);
      \draw[ultra thick, gray] (F)--(I);
      \draw[ultra thick, gray] (E)--(G);
      \draw[ultra thick, gray] (G)--(I);
      \draw[ultra thick, gray] (H)--(I);
      \draw[ultra thick, gray] (E)--(H);
      \draw[ultra thick, dashed, gray] (C)--(H);
      \draw[ultra thick, dashed, gray] (A)--(C);
      \draw[ultra thick, dashed, gray] (C)--(F);

      
      \tkzDefMidPoint(A,B) \tkzGetPoint{a}
      \tkzDefMidPoint(A,C) \tkzGetPoint{b}
      \tkzDefMidPoint(A,E) \tkzGetPoint{c}
      \tkzDefMidPoint(A,D) \tkzGetPoint{axb}

      
      \tkzMarkAngle(E,A,D)
      \tkzLabelAngle[pos = .6](E,A,D){$\theta$}
%% \tkzMarkRightAngle(C,E,O)

      \node[left,penColor4] at (c) {$\vec{c}$};
      \node[above,penColor2] at (b) {$\vec{b}$};
      \node[below,penColor] at (a) {$\vec{a}$};
      \node[left,penColor3] at (axb) {$\vec{a}\cross\vec{b}$};
      %% \tkzLabelPoints[above](c)
      %% \tkzLabelPoints[below](b)
      %% \tkzLabelPoints[left](a) {$\vec{a}$}
    \end{tikzpicture}
  \end{image}
    and recall that the volume of a parallelpiped is given by
    \[
    \text{area of the base}\times\text{height}
    \]
    to give a geometric verification of the \textbf{two} facts:
    \begin{itemize}
    \item First, that
      \[
      \left |\det\begin{bmatrix}
      a_1 & a_2 & a_3\\
      b_1 & b_2 & b_3\\
      c_1 & c_2 & c_3
      \end{bmatrix}\right|
      \]
      is the volume of the parallelepiped spanned by vectors $\vec{a}$,
      $\vec{b}$, and $\vec{c}$.
    \item Second, that
      \[
      \vec{a}\dotp(\vec{b}\cross\vec{c}) =
      (\vec{a}\cross\vec{b})\dotp\vec{c}
      \]
    \end{itemize}
\end{problem}

\end{document}
