\documentclass[noauthor,nooutcomes]{ximera}

\input{../../../preamble.tex}

\author{Austin Antoniou and Bart Snapp}

\title[Collaborate:]{Line integrals}

\begin{document}
\begin{abstract}
  We work some examples of line integrals and vector fields.
\end{abstract}
\maketitle

\textbf{Work in groups of 3--4, writing your answers on a separate
  sheet of paper.}

\begin{problem}
  Carefully sketch the following vector fields:
  \begin{enumerate}
  \item $\vec{F}(x,y) = \vector{1,1}$
  \item $\vec{G}(x,y) = \vector{x,y}$
  \item $\vec{H}(x,y) = \vector{y,-x}$
  \item $\vec{I}(x,y) = \vector{x,-y}$
  \end{enumerate}
\end{problem}

\begin{problem}
  Let $C$ be the circle of radius $1$, centered at the origin, drawn
  once in a counterclockwise direction. Using the fields above,
  compute:
  \begin{enumerate}
  \item $\int_C \vec{F}\dotp \d \vec{p}$
  \item $\int_C \vec{G}\dotp \d \vec{p}$
  \item $\int_C \vec{H}\dotp \d \vec{p}$
  \item $\int_C \vec{I}\dotp \d \vec{p}$
  \end{enumerate}
\end{problem}

\begin{problem}
  Let $T$ be the triangle with vertices at $(-2,-1)$, $(2,-1)$, and
  $(0,2)$ drawn once in a counterclockwise direction. Using the fields
  above, compute:
  \begin{enumerate}
  \item $\int_T \vec{F}\dotp \d \vec{p}$
  \item $\int_T \vec{G}\dotp \d \vec{p}$
  \item $\int_T \vec{H}\dotp \d \vec{p}$
  \item $\int_T \vec{I}\dotp \d \vec{p}$
  \end{enumerate}
\end{problem}

\begin{problem}
  Let $A$ be the path connecting $(-2,-1)$ to $(2,-1)$ to
  $(0,2)$. Note, this is \textbf{not} a closed path. Using the fields
  above, compute:
  \begin{enumerate}
  \item $\int_A \vec{F}\dotp \d \vec{p}$
  \item $\int_A \vec{G}\dotp \d \vec{p}$
  \item $\int_A \vec{H}\dotp \d \vec{p}$
  \item $\int_A \vec{I}\dotp \d \vec{p}$
  \end{enumerate}
\end{problem}


\begin{definition}
  The \dfn{work} $W$ done by a force $\vec{F}$ on an object moving
  along a curve $C$, parameterized by $\vec{p}(t)$ is given by:
  \[
  W = \int_C \vec{F}\dotp \d \vec{p}
  \]
\end{definition}


\begin{problem}
  A raindrop of mass $0.001\unit{kg}$ slides from the top of a flat
  windshield to the bottom. If the windshield is $\frac{1}{2} \unit{m}$
  high and $\frac{1}{2}\unit{m}$ deep, how much work is done by
  gravity? (Assume that the acceleration due to gravity is $10
  \unit{m}/\unit{s}^2$.)
\end{problem}

\begin{problem}
  A skateboarder of mass $70\unit{kg}$ rolls from the left side of a
  circular half-pipe of radius $3\unit{m}$ to the right side. How much
  work is done by gravity?
  \begin{image}
    \begin{tikzpicture}
      \draw[ultra thick,penColor] (.5,3.5)--(0,3.5)--(0,0)--(7,0)--(7,3.5)--(6.5,3.5);
      \draw[ultra thick,penColor] (.5,3.5) arc (180:360:3);
    \end{tikzpicture}
  \end{image}
\end{problem}


\end{document}
