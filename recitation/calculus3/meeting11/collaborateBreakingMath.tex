\documentclass[noauthor,nooutcomes]{ximera}

\input{../../../preamble.tex}

\author{Bart Snapp}

\title[Collaborate:]{Breaking math}

\begin{document}
\begin{abstract}
  Two calculus students attempt to ``break'' math.
\end{abstract}
\maketitle

\textbf{Work in groups of 3--4, writing your answers on a separate
  sheet of paper.}


Check out this dialogue between two calculus students (based on a true
story):

\begin{dialogue}
\item[Devyn] Riley, I have something very important to say.
\item[Riley] Yeah?  Hit me with it.
\item[Devyn] I think I just broke math.
\item[Riley] I've suspected for ages that all this calculus stuff was razzmatazz. Lay it on me.
\item[Devyn] Consider the vector field:
  \[
  \vector{\frac{-y}{x^2+y^2},\frac{x}{x^2+y^2}}
  \]
\item[Riley] Got it. It looks like a whirlpool.
\item[Devyn] I know! Now compute its curl.
\item[Riley] OK--I get zero curl.
\item[Devyn] I know! Now consider Green's Theorem.
\item[Riley] You mean:
  \[
  \iint_R \curl \vec{F} \d A = \oint_{C} \vec{F}\dotp\d\vec{p}
  \]
  What's $R$? What's $C$?
\item[Devyn] Let $R$ be the unit disk centered at the origin.
\item[Riley] OK, so $C$ is the unit circle centered at the origin.
\item[Devyn] Right. Now here's the deal\dots The left-hand side of the
  equation is zero because the curl of our vector field is zero.
\item[Riley] Oh. And the right-hand side of the equation cannot be
  zero because our vector field looks like a whirlpool.
\item[Devyn] And now we have zero equals something not zero, and kablamy, math is in ruins.
\end{dialogue}


\begin{problem}
  What just happened? Explain why Devin and Riley think math is
  broken. Try to take it step-by-step.
\end{problem}

Let's investigate further. To really understand this, we're going to
have to check each of their claims.

\begin{problem}
  Consider the vector field
  \[
  \vector{\frac{-y}{x^2+y^2},\frac{x}{x^2+y^2}}
  \]
  Plot some vectors on the grid below. Focus on getting the directions
  of the vectors correct, and don't worry to much about the
  magnitudes.
  \begin{image}
    \begin{tikzpicture}
      \begin{axis}%
        [
	  ymin=-4,ymax=4,
	  xmin=-4,xmax=4,
          axis lines =middle, xlabel=$x$, ylabel=$y$,
          every axis y label/.style={at=(current axis.above origin),anchor=south},
          every axis x label/.style={at=(current axis.right of origin),anchor=west},
          grid=both,
          grid style={dashed, gridColor},
          xtick={-6,...,6},
          ytick={-6,...,6},
	]
      \end{axis}
     \end{tikzpicture}
  \end{image}
  Does it look like a ``whirlpool?''
\end{problem}

\begin{problem}
  Letting $\vec{F}(x,y) = \vector{\frac{-y}{x^2+y^2},\frac{x}{x^2+y^2}}$,
  explain why someone might believe that
  \[
  \oint_{C} \vec{F}\dotp\d\vec{p}
  \]
  is nonzero. No computation are necessary at this point.
\end{problem}


\begin{problem}
  Parametrize the unit circle $U$ centered at the origin and compute:
  \[
  \oint_{U} \vec{F}\dotp\d\vec{p}
  \]
\end{problem}

\begin{problem}
  Parametrize a circle $C$ of radius $r$ centered at the origin and
  compute:
  \[
  \oint_{C} \vec{F}\dotp\d\vec{p}
  \]
\end{problem}

\begin{problem}
    As a gesture of friendship, we reveal that:
  \begin{align*}
    \grad \arctan(y/x) &= \vector{\frac{-y}{x^2+y^2},\frac{x}{x^2+y^2}}\\
    \grad \arctan(-x/y) &= \vector{\frac{-y}{x^2+y^2},\frac{x}{x^2+y^2}}
  \end{align*}
  Confirm these equations.
\end{problem}

\begin{problem}
  Use the Fundamental Theorem of Line Integrals to compute
  \[
  \oint_S \frac{-y}{x^2+y^2}\d x + \frac{x}{x^2+y^2}\d y
  \]
  where $S$ is the square with vertices $(1,1)$, $(-1,1)$, $(-1,-1)$,
  and $(1,-1)$ drawn in a counterclockwise fashion.
  \begin{hint}
    For the fundamental theorem to apply, the chosen path must be in
    the domain of the potential function.
  \end{hint}
\end{problem}

\begin{problem}
  Use the Fundamental Theorem of Line Integrals to compute
  \[
  \oint_Q \frac{-y}{x^2+y^2}\d x + \frac{x}{x^2+y^2}\d y
  \]
  where $Q$ is the square with vertices $(a,a)$, $(-a,a)$, $(-a,-a)$,
  and $(a,-a)$ drawn in a counterclockwise fashion, where $a>0$.
  \begin{hint}
    For the fundamental theorem to apply, the chosen path must be in
    the domain of the potential function.
  \end{hint}
\end{problem}

\begin{problem}
  Use the Fundamental Theorem of Line Integrals to compute
  \[
  \oint_Y \frac{-y}{x^2+y^2}\d x + \frac{x}{x^2+y^2}\d y
  \]
  where $Y$ is a polygonal path you choose for yourself that contains
  the point $(0,0)$ in the \textbf{interior}. A triangle of some sort
  would probably be easiest.
  \begin{hint}
    For the fundamental theorem to apply, the chosen path must be in
    the domain of the potential function.
  \end{hint}
\end{problem}

At this point, you might suspect that something strange is going
on\dots

\begin{problem}
  Again letting $\vec{F}(x,y) =
  \vector{\frac{-y}{x^2+y^2},\frac{x}{x^2+y^2}}$, compute:
  \[
  \curl \vec{F}
  \]
  Is $\curl \vec{F}$ zero?
  \begin{hint}
    It is not \textbf{always} zero.
  \end{hint}
\end{problem}

\begin{problem}
  What would you say to Devin and Riley to assure them that
  mathematics is not ``broken?''
\end{problem}


\newpage


\section{The take-away}

Here we presented you with a field where the (scalar) curl was zero
everywhere except at the origin.  At the origin the (scalar) curl was
undefined; hence, Green's Theorem does not apply.

What is remarkable is that in this case, where the (scalar) curl is
zero except for a point, any path $C$ around the point where the field
is undefined will yield the same value for:
\[
\oint_C \vec{F}\dotp\d\vec{p}
\]



\end{document}
