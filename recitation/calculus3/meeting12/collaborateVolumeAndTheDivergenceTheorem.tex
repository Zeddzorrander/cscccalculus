\documentclass[nooutcomes,noauthor]{ximera}

\author{Bart Snapp}

\input{../../../preamble.tex}

\title[Collaborate:]{Volume and the divergence theorem}

\begin{document}
\begin{abstract}
  We compute volumes using the divergence theorem.
\end{abstract}
\maketitle

\textbf{Work in groups of 3--4, writing your answers on a separate
  sheet of paper.}


So far, we've been using the divergence theorem to simplify the
computations of surface integrals. However, as we will see, we can
also use the divergence theorem to compute volumes of solid
regions. Specifically, we will now try to compute the volume of an
ellipsoid:
\[
\vec{E}(\theta,\phi) = \vector{a \cos(\theta)\sin(\phi), b \sin(\theta)\sin(\phi),c \cos(\phi)}
\]
for $0\le \theta< 2\pi$ and $0\le \phi\le \pi$.

\begin{problem}
  Give a careful sketch of the graph of $\vec{E}$:
  \begin{image}
    \begin{tikzpicture}[x=0.7cm,y=0.7cm,z=0.5cm]
      % The axes %%FAKED
      \draw[->] (xyz cs:x=-5) -- (xyz cs:x=5) node[above] {$y$};
      \draw[->] (xyz cs:y=-5) -- (xyz cs:y=5) node[right] {$z$};
      \draw[->] (xyz cs:z=5) -- (xyz cs:z=-5) node[above] {$x$};
      % The thin ticks
      \foreach \coo in {5,4,...,-5} %%x
               {
                 \draw (xyz cs:y=-0.15pt,z=\coo) -- (xyz cs:y=0.15pt,z=\coo);
               }
      \foreach \coo in {-5,-4,...,5}%% y
               {
                 \draw (\coo,-1.5pt) -- (\coo,1.5pt);
               }
      \foreach \coo in {-5,-4,...,5}
               {
                 \draw (-1.5pt,\coo) -- (1.5pt,\coo);
               }         
    \end{tikzpicture}
  \end{image}
\end{problem}

\begin{problem}
  Describe in pictures, words, interpretative dance, how the ellipsoid
  is drawn by $\vec{E}$ as $\theta$ runs from $0$ to $2\pi$, and
  $\phi$ runs from $0$ to $\pi$.
\end{problem}


\begin{problem}
  As a gesture of friendship, I will tell you that the implicit
  formula for an ellipsoid is:
  \[
  \frac{x^2}{a^2} + \frac{y^2}{b^2} + \frac{z^2}{c^2} =1
  \]
  Use the vector-valued formula $\vec{E}$ to confirm this formula.
\end{problem}

\begin{problem}
  If you were to set-up an iterated integral to compute the volume of
  the ellipsoid, which coordinates would be easiest to use? Why?
\end{problem}

\begin{problem}
  Set-up an iterated integral that will compute the volume of the
  ellipsoid.
\end{problem}

Now we will try to use the divergence theorem to compute volume. As a
gesture of friendship, we remind you of the statement of the
divergence theorem:
  \[
  \iiint_R \divergence \vec{F}  \d V =   \oiint_{\partial R} \vec{F}\dotp\uvec{n}\d S
  \]
The first thing we need are vector fields such that
$\divergence \vec{F} = 1$.

\begin{problem}
  Find $7$ (simple!) vector fields $\vec{F}: \R^3\to\R^3$ such that
  $\divergence\vec{F} =1$.
\end{problem}

\begin{problem}
  Setting $\vec{F} = \vector{x/3,y/3,z/3}$, use the divergence theorem
  to compute the volume of the ellipsoid.
\end{problem}



\end{document}
